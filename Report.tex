\documentclass{article}
\usepackage[utf8]{inputenc}
\usepackage{amssymb}
\usepackage{amsmath}
\usepackage{float}
\usepackage{epstopdf}
\usepackage{moreverb}
\usepackage{multicol}
\usepackage{listings}
\usepackage{mathrsfs}
\usepackage{graphicx}
\usepackage{cite}
\usepackage{tabularx}
\usepackage{listings}
\newcommand{\R}{\mathbb{R}}
\newcommand{\overbar}[1]{\mkern 1.5mu\overline{\mkern-1.5mu#1\mkern-1.5mu}\mkern 1.5mu}

\newcommand{\pdiff}[2]{\frac{\partial {#1}}{\partial {#2}}}



\pagenumbering{gobble}
\bibliographystyle{plain}
\title{APPM 5720 Homework 2}
\author{Will Boshell, Fortino Garcia, Alex Rybchuk, Parth Thakkar}
\date{January 31, 2018}


\begin{document}
%\def\code#1{\texttt{#1}}

\maketitle

\section{Newton's Method}
The code in \verb|newtonS.p| and \verb|newtonS.f90.Template| performed Newton's method and a set of simple sample functions
\begin{align*}
	& f\left( x \right) = e^x - 1 \, , \quad g\left( x \right) = x^2\\
	& h\left( x \right) = \sin(x) \, , \quad r \left( x \right) = e^{-1/x}\\
\end{align*}
where each function has a root at $x = 0$, save for $r\left( x \right)$ which is undefined at $0$ though
\begin{align*}
	\lim_{x \to 0^+} r(x) = \lim_{x \to 0^+} e^{-1/x} = 0
\end{align*}

We recall that Newton's method is quadratically convergent to a root, under the assumption that the root is simple and the derivative is itself not $0$ at the root of interest (and the second derivative is bounded). Let us examine each function individually.

\subsection{$e^x - 1$}
Here is some discussion.
\begin{table}[H]
	\begin{tabularx}{1\textwidth}{ |>{\setlength\hsize{0.5\hsize}\centering}X| >{\setlength\hsize{1.25\hsize}\centering}X|>{\setlength\hsize{1.25\hsize}\centering}X| } 
	  \hline
	Iteration & $$E_{n+1}/E_{n}$$ & $$E_{n+1}/E_{n}^2$$\tabularnewline
	\hline 
	 01 & -0.6321205588285577E+00 & -0.6321205588285577E+00 \tabularnewline
	\hline 
	 02 & 0.4869314375964384E+00 & -0.7703141921199606E+00 \tabularnewline
	\hline 
	 03 & 0.1894444060136254E+00 & -0.6154801567949590E+00 \tabularnewline
	\hline 
	 04 & 0.3031399383262888E-01 & -0.5198686660785536E+00 \tabularnewline
	\hline 
	 05 & 0.8848598680903540E-03 & -0.5005896024729687E+00 \tabularnewline
	\hline 
	 06 & 0.7820703579018673E-06 & -0.5000099574951582E+00 \tabularnewline
	\hline 
	 07 & 0.0000000000000000E+00 & -0.0000000000000000E+00 \tabularnewline
	\hline 
	\end{tabularx}
\end{table}
\noindent Maybe post discussion.
\newpage
%%%%%%%%%%%%%%%%%%%%%%%%%%%%%%%%%%%%%%%
%%%%%%%%%%%%%%%%%%%%%%%%%%%%%%%%%%%%%%%

\subsection{$f(x) = x^2$}
We can see that this function has a double root at $x=0$. The derivative of f(x) is 0 at $x=0$. Thus, using the Newton's method, we would expect linear convergence instead of quadratic convergence. Implementation of Newton's method shows that the ratio of errors from adjecent time steps, i.e. $E_{n+1}/E_{n}$ is a constant, i.e. the convergence is linear.
\begin{table}[H]
	\centering
	\begin{tabularx}{1\textwidth}{ |>{\setlength\hsize{0.5\hsize}\centering}X| >{\setlength\hsize{1.25\hsize}\centering}X|>{\setlength\hsize{1.25\hsize}\centering}X| }
	  \hline
	Iteration & $$E_{n+1}/E_{n}$$ & $$E_{n+1}/E_{n}^2$$\tabularnewline
	\hline 
	 01 & -0.5000000000000000E+00 & -0.5000000000000000E+00 \tabularnewline
	\hline 
	 02 & 0.5000000000000000E+00 & -0.1000000000000000E+01 \tabularnewline
	\hline 
	 03 & 0.5000000000000000E+00 & -0.2000000000000000E+01 \tabularnewline
	\hline 
	 04 & 0.5000000000000000E+00 & -0.4000000000000000E+01 \tabularnewline
	\hline 
	 05 & 0.5000000000000000E+00 & -0.8000000000000000E+01 \tabularnewline
	\hline 
	 06 & 0.5000000000000000E+00 & -0.1600000000000000E+02 \tabularnewline
	\hline 
	 07 & 0.5000000000000000E+00 & -0.3200000000000000E+02 \tabularnewline
	\hline 
	 08 & 0.5000000000000000E+00 & -0.6400000000000000E+02 \tabularnewline
	\hline 
	 09 & 0.5000000000000000E+00 & -0.1280000000000000E+03 \tabularnewline
	\hline 
	 10 & 0.5000000000000000E+00 & -0.2560000000000000E+03 \tabularnewline
	\hline 
	 11 & 0.5000000000000000E+00 & -0.5120000000000000E+03 \tabularnewline
	\hline 
	 12 & 0.5000000000000000E+00 & -0.1024000000000000E+04 \tabularnewline
	\hline 
	 13 & 0.5000000000000000E+00 & -0.2048000000000000E+04 \tabularnewline
	\hline 
	 14 & 0.5000000000000000E+00 & -0.4096000000000000E+04 \tabularnewline
	\hline 
	 15 & 0.5000000000000000E+00 & -0.8192000000000000E+04 \tabularnewline
	\hline 


	\end{tabularx}
\end{table}

\noindent The Newton's method calculates the root after the $(k+1)^{th}$ iteration by the iterative formula
\begin{equation}
    x_{n+1} = x_{n} - \frac{f(x_{n}}{f'(x_{n})}
\end{equation}
Since we know that the root is a double root, we try modifying the formula by multiplying a factor of 2 to the 2nd term, i.e.
\begin{equation}
    x_{n+1} = x_{n} - 2 \frac{f(x_{n}}{f'(x_{n})}
\end{equation}
This iterative formula ends up converging to  the root $x=0$ immediately on the first iteration irrespective of our initial guess! Say our initial guess is $x=x_{0}$. Then the first iteration would result in
\begin{align*}
    x_{1} &= x_{0} - 2 \frac{f(x_{0}}{f'(x_{0})}\\
    &= x_{0} - 2 \frac{x_{0}^{2}}{2 x_{0}}\\
    &= 0
\end{align*}
as demonstrated by the code.
\begin{table}[H]
	\centering
	\begin{tabularx}{1\textwidth}{ |>{\setlength\hsize{0.5\hsize}\centering}X| >{\setlength\hsize{1.25\hsize}\centering}X|>{\setlength\hsize{1.25\hsize}\centering}X| } 
	  \hline
	Iteration & $$E_{n+1}/E_{n}$$ & $$E_{n+1}/E_{n}^2$$\tabularnewline
	\hline 
	 01 & -0.1000000000000000E+01 & -0.1000000000000000E+01 \tabularnewline
	\hline 
	\end{tabularx}
\end{table}]
%%%%%%%%%%%%%%%%%%%%%%%%%%%%%%%%%%%%%%%
%%%%%%%%%%%%%%%%%%%%%%%%%%%%%%%%%%%%%%%

\newpage
%%%%%%%%%%%%%%%%%%%%%%%%%%%%%%%%%%%%%%%
%%%%%%%%%%%%%%%%%%%%%%%%%%%%%%%%%%%%%%%

\subsection{$\sin (x)$}
\begin{table}[H]
	\centering
	\begin{tabularx}{1\textwidth}{ |>{\setlength\hsize{0.5\hsize}\centering}X| >{\setlength\hsize{1.25\hsize}\centering}X|>{\setlength\hsize{1.25\hsize}\centering}X| } 
	  \hline
	Iteration & $$E_{n+1}/E_{n}$$ & $$E_{n+1}/E_{n}^2$$\tabularnewline
	\hline 
	 01 & -0.1523676741017902E+01 & -0.1523676741017902E+01 \tabularnewline
	\hline 
	 02 & -0.3877900367215199E+00 & 0.2545093892175937E+00 \tabularnewline
	\hline 
	 03 & -0.9689555030589873E-01 & -0.1639888607225156E+00 \tabularnewline
	\hline 
	 04 & -0.1090466714959705E-02 & 0.1904667166963419E-01 \tabularnewline
	\hline 
	 05 & -0.1299242582583448E-08 & -0.2081059578567191E-04 \tabularnewline
	\hline 
	 06 & 0.0000000000000000E+00 & -0.0000000000000000E+00 \tabularnewline
	\hline 
	\end{tabularx}
\end{table}

\noindent But here is one with the cubic 
\begin{table}[H]
	\centering
	\begin{tabularx}{1\textwidth}{ |>{\setlength\hsize{0.5\hsize}\centering}X| >{\setlength\hsize{1.25\hsize}\centering}X|>{\setlength\hsize{1.25\hsize}\centering}X| } 
	  \hline
	Iteration & $$E_{n+1}/E_{n}$$ & $$E_{n+1}/E_{n}^3$$\tabularnewline
	\hline 
	 01 & -0.1523676741017902E+01 & -0.1523676741017902E+01 \tabularnewline
	\hline 
	 02 & -0.3877900367215199E+00 & -0.1670363420049105E+00 \tabularnewline
	\hline 
	 03 & -0.9689555030589873E-01 & -0.2775395398051782E+00 \tabularnewline
	\hline 
	 04 & -0.1090466714959705E-02 & -0.3326792984270507E+00 \tabularnewline
	\hline 
	 05 & -0.1299242582583448E-08 & -0.3333333611137316E+00 \tabularnewline
	\hline 
	 06 & 0.0000000000000000E+00 & 0.0000000000000000E+00 \tabularnewline
	\hline 
\end{tabularx}
\end{table}
\noindent Further.
%%%%%%%%%%%%%%%%%%%%%%%%%%%%%%%%%%%%%%%
%%%%%%%%%%%%%%%%%%%%%%%%%%%%%%%%%%%%%%%

\newpage
%%%%%%%%%%%%%%%%%%%%%%%%%%%%%%%%%%%%%%%
%%%%%%%%%%%%%%%%%%%%%%%%%%%%%%%%%%%%%%%

\subsection{$e^{-1/x}$}

\begin{table}[htbp]
	\centering
	\begin{tabularx}{1\textwidth}{ |>{\setlength\hsize{0.5\hsize}\centering}X| >{\setlength\hsize{1.25\hsize}\centering}X|>{\setlength\hsize{1.25\hsize}\centering}X| } 
	  \hline
	Iteration & $$E_{n+1}/E_{n}$$ & $$E_{n+1}/E_{n}^2$$\tabularnewline
	\hline 
	 01 & -0.1000000000000000E-03 & -0.1000000000000000E-03 \tabularnewline
	\hline 
	 02 & 0.9801000000000002E+00 & -0.9801000000000002E+04 \tabularnewline
	\hline 
	 03 & 0.9802980099999998E+00 & -0.1000202030405060E+05 \tabularnewline
	\hline 
	 04 & 0.9804920990079601E+00 & -0.1020506060403061E+05 \tabularnewline
	\hline 
	 05 & 0.9806823827241055E+00 & -0.1041012070006161E+05 \tabularnewline
	\hline 
	 06 & 0.9808689723299778E+00 & -0.1061720039624038E+05 \tabularnewline
	\hline 
	 07 & 0.9810519747073182E+00 & -0.1082629950052112E+05 \tabularnewline
	\hline 
	 08 & 0.9812314926443280E+00 & -0.1103741782460229E+05 \tabularnewline
	\hline 
	 09 & 0.9814076250301539E+00 & -0.1125055518381782E+05 \tabularnewline
	\hline 
	 10 & 0.9815804670383682E+00 & -0.1146571139703251E+05 \tabularnewline
	\hline 
	 11 & 0.9817501103001732E+00 & -0.1168288628654150E+05 \tabularnewline
	\hline 
	 12 & 0.9819166430679892E+00 & -0.1190207967797336E+05 \tabularnewline
	\hline 
	 13 & 0.9820801503700450E+00 & -0.1212329140019697E+05 \tabularnewline
	\hline 
	 14 & 0.9822407141565502E+00 & -0.1234652128523173E+05 \tabularnewline
	\hline 
	 15 & 0.9823984134379694E+00 & -0.1257176916816113E+05 \tabularnewline
	\hline 
\end{tabularx}
\end{table}
%%%%%%%%%%%%%%%%%%%%%%%%%%%%%%%%%%%%%%%
%%%%%%%%%%%%%%%%%%%%%%%%%%%%%%%%%%%%%%%

\end{document}